\chapter{Distributed Erlang}
\label{distribution}

A \textbf{distributed Erlang system} consists of a number of Erlang
runtime systems communicating with each other. Each such runtime
system is called a \textbf{node}. Nodes can reside on the same
processor or on different processors connected through a network. The
standard distribution mechanism is implemented using TCP/IP sockets
but other mechanisms could also be implemented.

Message passing between processes at different nodes, as well as links
and monitors, is transparent when using Pids. However, registered
names are local to each node. A registered process at a particular
node is referred to as \texttt{\{Name,Node\}}.

The Erlang Port Mapper Daemon \textbf{epmd} is automatically started
at every host where an Erlang node is started. It is responsible for
mapping the symbolic node names to machine addresses.


\section{Nodes}
A \textbf{node} is an executing Erlang runtime system which has been
given a name, using the command line flag \texttt{-name} (long names)
or \texttt{-sname} (short names).

The format of the node name is an atom \texttt{Name@Host} where
\texttt{Name} is the name given by the user and \texttt{Host} is the
full host name if long names are used, or the first part of the host
name if short names are used. \texttt{node()} returns the name of the
node. Nodes with long node names cannot communicate with nodes with
short node names.


\section{Node connections}
The nodes in a distributed Erlang system are loosely connected. The
first time the name of another node is used, a connection attempt to
that node will be made. If a node A connects to node B, and node B has
a connection to node C, then node A will also try to connect to node
C. This feature can be turned off by the command line flag:

\begin{erlang}
-connect_all false.
\end{erlang}


If a node goes down, all connections to that node are removed. The
BIF:

\begin{erlang}
erlang:disconnect(Node)
\end{erlang}

disconnects Node. The BIF \texttt{nodes()} returns the list of
currently connected (visible) nodes.


\section{Hidden nodes}
It is sometimes useful to connect to a node without also connecting to
all other nodes. For this purpose, a \textbf{hidden node} may be
used. A hidden node is a node started with the command line flag
-hidden. Connections between hidden nodes and other nodes must be set
up explicitly. Hidden nodes do not show up in the list of nodes
returned by nodes(). Instead, nodes(hidden) or nodes(connected) must
be used. A hidden node will not be included in the set of nodes that
the module global keeps track of.

A \textbf{C node} is a C program written to act as a hidden node in a
distributed Erlang system. The library \texttt{erl\_interface}
contains functions for this purpose.


\section{Cookies}
Each node has its own \textbf{magic cookie}, which is an atom. The
Erlang network authentication server (auth) reads the cookie in the
file \texttt{\$HOME/.erlang.cookie}. If the file does not exist, it
will be created with contents as a random string.

The UNIX permissions mode of the file is set to octal 400 (read-only
by user). The cookie of the local node is set using the BIF
\texttt{erlang:set\_cookie(node(), Cookie)}.

The current node is only allowed to communicate with another node
\texttt{Node2} with a different cookie \texttt{Cookie2} if it knows
the value of this cookie and calls the BIF
\texttt{erlang:set\_cookie(Node2, Cookie2)}.


\section{Distribution BIFs}

\begin{center}
\begin{tabular}{|>{\raggedright}p{156pt}|>{\raggedright}p{170pt}|}
\hline
\multicolumn{2}{|p{326pt}|}{Distribution BIFs}\tabularnewline
\hline
\texttt{node()}  & Returns the name of the current node. Allowed in guards\tabularnewline
\hline
\texttt{is\_alive()}  & Returns true if the runtime system is a node and can connect to
other nodes, false otherwise\tabularnewline
\hline
\texttt{erlang:get\_cookie()}  & Returns the magic cookie of the current node\tabularnewline
\hline
\texttt{set\_cookie(Node, Cookie)} & Sets the magic cookie used when connecting to Node.
If Node is the current node, Cookie will be used when connecting to all new nodes\tabularnewline
\hline
\texttt{nodes()}  & Returns a list of all visible nodes that the current node is connected
to\tabularnewline
\hline
\texttt{nodes(connected \textbar{} hidden)}  & Returns a list not only of visible nodes,
but also hidden nodes and previously known nodes, etc. \tabularnewline
\hline
\texttt{monitor\_node(Node, true\textbar{}false)}  & Monitors the status of Node. A message
\texttt{\{nodedown, Node\}} is received if the connection to it is lost\tabularnewline
\hline
\texttt{node(Pid\textbar{}Ref\textbar{}Port)}  & Returns the node where the argument is
located\tabularnewline
\hline
\texttt{erlang:disconnect\_node(Node)}  & Forces the disconnection of a node\tabularnewline
\hline
\texttt{spawn[\_link\textbar{}\_opt](Node, Module, Function, Args)}  & Creates a process
at a remote node\tabularnewline
\hline
\texttt{spawn[\_link\textbar{}\_opt](Node, Fun)}  & Creates a process at a remote node\tabularnewline
\hline
\end{tabular}
\end{center}


\section{Distribution command line flags}

\begin{center}
\begin{tabular}{|>{\raggedright}p{102pt}|>{\raggedright}p{224pt}|}
\hline
\multicolumn{2}{|p{326pt}|}{Distribution command line flags}\tabularnewline
\hline
\texttt{-connect\_all false}  & Only explicit connection set-ups will be used. \tabularnewline
\hline
\texttt{-hidden}  & Makes a node into a hidden node. \tabularnewline
\hline
\texttt{-name Name}  & Makes a runtime system into a node, using long node names. \tabularnewline
\hline
\texttt{-setcookie Cookie}  & Same as calling \linebreak{}
\texttt{erlang:set\_cookie(node(), Cookie))}\tabularnewline
\hline
\texttt{-sname Name}  & Makes a runtime system into a node, using short node names. \tabularnewline
\hline
\end{tabular}
\end{center}


\section{Distribution modules}
There are several modules available useful for distributed programming.

\begin{center}
\begin{tabular}{|>{\raggedright}p{93pt}|>{\raggedright}p{233pt}|}
\hline
\multicolumn{2}{|p{326pt}|}{Kernel modules useful for distribution}\tabularnewline
\hline
\texttt{global}  & A global name registration facility\tabularnewline
\hline
\texttt{global\_group}  & Grouping nodes to global name registration groups\tabularnewline
\hline
\texttt{net\_adm}  & Various net administration routines\tabularnewline
\hline
\texttt{net\_kernel}  & Erlang networking kernel\tabularnewline
\hline
\multicolumn{2}{|p{326pt}|}{S{\large{}TDLIB modules useful for distribution}}\tabularnewline
\hline
\texttt{slave}  & Start and control of slave nodes\tabularnewline
\hline
\end{tabular}
\end{center}