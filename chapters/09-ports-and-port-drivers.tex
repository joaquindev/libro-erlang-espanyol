\chapter{Ports and Port Drivers}
\label{ports}
\textbf{Ports} provide a byte-oriented interface to external programs
and communicate with Erlang processes by sending and receiving lists
of bytes as messages. The Erlang process that creates a port is called
the \textbf{port owner} or the \textbf{connected process} of the
port. All communication to and from the port should go via the port
owner. If the port owner terminates, so will the port (and the
external program, if it has been programmed correctly).

The external program forms another OS process. By default, it should
read from standard input (file descriptor 0) and write to standard
output (file descriptor 1). The external program should terminate when
the port is closed.


\section{Port Drivers}
Drivers are normally programmed in C and are dynamically linked to the
Erlang runtime system. The linked-in driver behaves like a port and is
called a \textbf{port driver}. However, an erroneous port driver might
cause the entire Erlang runtime system to leak memory, hang or crash.


\section{Port BIFs}

\begin{center}
\begin{tabular}{|>{\raggedright}p{161pt}|>{\raggedright}p{165pt}|}
\hline
\multicolumn{2}{|p{326pt}|}{Port creation BIF}\tabularnewline
\hline
\texttt{open\_port(PortName, PortSettings)} & Returns a \textbf{port identifier} \texttt{Port} as the result of opening a new Erlang port. Messages can be sent to and received from a port identifier, just like a Pid. Port identifiers can also be linked to or registered under a name using \texttt{link/1} and \texttt{register/2}. \tabularnewline
\hline
\end{tabular}
\end{center}

\texttt{PortName} is usually a tuple \texttt{\{spawn,Command\}} where
the string \texttt{Command} is the name of the external program. The
external program runs outside the Erlang workspace unless a port
driver with the name \texttt{Command} is found. If the driver is
found, it will be started.

\texttt{PortSettings} is a list of settings (options) for the
port. The list typically contains at least a tuple
\texttt{\{packet,N\}} which specifies that data sent between the port
and the external program are preceded by an N-byte length
indicator. Valid values for \texttt{N} are 1, 2 or 4. If binaries
should be used instead of lists of bytes, the option \texttt{binary}
must be included.

The port owner Pid communicates with Port by sending and receiving
messages. (Any process could send the messages to the port, but
messages from the port will always be sent to the port owner).

\begin{center}
\begin{tabular}{|>{\raggedright}p{115pt}|>{\raggedright}p{211pt}|}
\hline
\multicolumn{2}{|p{326pt}|}{Messages sent to a port}\tabularnewline
\hline
\texttt{\{Pid, \{command, Data\}\}}  & Sends Data to the port. \tabularnewline
\hline
\texttt{\{Pid, close\}}  & Closes the port. Unless the port is already closed, the port replies
with \texttt{\{Port, closed\}} when all buffers have been flushed and the port really closes.
\tabularnewline
\hline
\texttt{\{Pid,\{connect,NewPid\}\}}  & Sets the port owner of \texttt{Port} to \texttt{NewPid}. Unless the port is already closed, the port replies with \texttt{\{Port, connected\}} to the old port owner. Note that the old port owner is still linked to the port, but the new port
owner is not. \tabularnewline
\hline
\end{tabular}
\end{center}

Data must be an I/O list. An I/O list is a binary or a (possibly deep)
list of binaries or integers in the range 0..255.

\begin{center}
\begin{tabular}{|>{\raggedright}p{121pt}|>{\raggedright}p{204pt}|}
\hline
\multicolumn{2}{|p{326pt}|}{Messages received from a port}\tabularnewline
\hline
\texttt{\{Port, \{data, Data\}\}}  & Data is received from the external program\tabularnewline
\hline
\texttt{\{Port, closed\}}  & Reply to \texttt{Port ! \{Pid,close\}}\tabularnewline
\hline
\texttt{\{Port, connected\}}  & Reply to \texttt{Port ! \{Pid,\{connect, NewPid\}\}} \tabularnewline
\hline
\texttt{\{'EXIT', Port, Reason\}}  & If Port has terminated for some reason. \tabularnewline
\hline
\end{tabular}
\end{center}

Instead of sending and receiving messages, there are also a number of
BIFs that can be used. These can be called by any process, not only
the port owner.

\begin{center}
\begin{tabular}{|>{\raggedright}p{146pt}|>{\raggedright}p{180pt}|}
\hline
\multicolumn{2}{|p{326pt}|}{Port BIFs}\tabularnewline
\hline
\texttt{port\_command(Port, Data)}  & Sends Data to Port\tabularnewline
\hline
\texttt{port\_close(Port)}  & Closes Port\tabularnewline
\hline
\texttt{port\_connect(Port, NewPid)}  & Sets the port owner of \texttt{Port} to \texttt{NewPid}. The old port owner Pid stays linked to the port and has to call \texttt{unlink(Port)} if this is not desired. \tabularnewline
\hline
\texttt{erlang:port\_info(Port, Item)}  & Returns information as specified by \texttt{Item}\tabularnewline
\hline
\texttt{erlang:ports()}  & Returns a list of all ports on the current node\tabularnewline
\hline
\end{tabular}
\end{center}

There are some additional BIFs that only apply to port drivers:
\texttt{port\_control/3} and \texttt{erlang:port\_call/3}.