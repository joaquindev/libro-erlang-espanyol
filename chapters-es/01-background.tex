\chapter{Antecedentes del lenguaje Erlang}
\label{background}

Erlang es el resultado de un proyecto del Laboratorio de inform\'atica de Ericsson para mejorar la programaci\'on 
de las aplicaciones en el campo de las telecomunicaciones. Un requisito fundamental era soportar las caracter\'isticas
que describen a dichas aplicaciones, las cuales incluyen:

\begin{itemize}
\item Concurrencia masiva

\item Tolerancia a fallos

\item Aislamiento

\item Actualizaci\'on de c\'odigo din\'amica en tiempo de ejecuci\'on

\item Transacciones
\end{itemize}

A lo largo de la historia de Erlang el proceso de desarrollo ha sido extremadamente pragm\'atico. Las 
caracter\'isticas y propiedades de los tipos de sistemas en los que Ericsson estaban interesados
fue la principal influencia en el desarrollo de Erlang. Estas propiedades se consideraron tan fundamentales
que se decidi\'o construir el soporte para ellas en el propio lenguaje, en lugar de necesitar bibliotecas.
Debido al proceso de desarrollo pragm\'atico del lenguaje, Erlang se convirti\'o en un lenguaje funcional
sin haberlo previsto --- dado que las caracter\'isticas de los lenguajes funcionales encajaban de forma 
adecuada en los sistemas que se estaban desarrollando.
