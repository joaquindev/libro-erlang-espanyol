\chapter{Estructura de un programa en Erlang}

\section{Sintaxis de los m\'odulos}

Un programa de Erlang est\'a compuesto \textbf{por m\'odulos} donde cada m\'odulo
es un fichero de texto con extensi\'on \textbf{.erl}. En programas peque\~nos, todos
los m\'odulos residen en el mismo directorio. Un m\'odulo consiste de atributos y 
definici\'on de funciones.

\begin{erlang}
-module(demo).
-export([double/1]).

double(X) -> times(X, 2).

times(X, N) -> X * N.
\end{erlang}

El m\'odulo \texttt{demo} de arriba consiste en la funci\'on \texttt{times/2}
y es local al m\'odulo y de la funci\'on \texttt{double/1} que se exporta
y que puede ser llamada desde fuera del m\'odulo.

\texttt{demo:double(10)} \resultingin \texttt{20}\hfill
(la flecha $\Rightarrow$ se lee como ``da como resultado'') \texttt{double/1} significa la funci\'on ``double'' con \textit{one}(un)
argumento. Una funci\'on \texttt{double/2} que recibe \textit{two}(dos) argumentos se considera una funci\'on diferente. El n\'umero de argumentos se llama la \textbf{aridad} de la funci\'on.

\section{Atributos de m\'odulo}
Un \textbf{atributo de un m\'odulo} define una propiedad concreta de un m\'odulo y consiste en una \textbf{etiqueta}(tag) y un \textbf{valor}(value):
\texttt{-Tag(valor).} 

\texttt{Tag} tiene que se un \'atomo, mientras que \texttt{valor} tiene que ser un t\'ermino literal (consulta el cap\'itulo \ref{datatypes}). Cualquier atributo de un m\'odulo puede ser especificado. Los atributos se guardan en el c\'odigo compilado y puede ser recuperado llamando a la funci\'on \texttt{Module:module\_info(attributes).}

\subsection{Atributos de m\'odulo predefinidos}
Los atributos de m\'odulo predefinodos tienen que ponerse antes de cualquier declaraci\'on de funci\'on.

\begin{itemize}

	\item \begin{erlangim}
	-module(Module).
	\end{erlangim}
	This attribute is mandatory and must be specified first. It
        defines the name of the module. The name Module, an atom (see section \ref{datatypes:atom}),
        should be the same as the filename without the `\texttt{.erl}' extension.

	\item \begin{erlangim}
	-export([Func1/Arity1, ..., FuncN/ArityN]).
	\end{erlangim}
	This attribute specifies which functions in the module that
        can be called from outside the module. Each function name
        \texttt{FuncX} is an atom and \texttt{ArityX} an integer.

	\item \begin{erlangim}
	-import(Module,[Func1/Arity1, ..., FuncN/ArityN]).
	\end{erlangim}
	This attribute indicates a \texttt{Module} from which a list of functions
        are imported.  For example:

	\begin{erlangim}
	-import(demo, [double/1]).
	\end{erlangim}
	This means that it is possible to write \texttt{double(10)} instead of
        the longer \texttt{demo:double(10)} which can be impractical if the
        function is used frequently.

	\item \begin{erlangim}
	-compile(Options).
	\end{erlangim}
	Compiler options.

	\item \begin{erlangim}
	-vsn(Vsn).
	\end{erlangim}
	Module version. If this attribute is not specified, the
        version defaults to the checksum of the module.

	\item \begin{erlangim}
	-behaviour(Behaviour).
	\end{erlangim}
	This attribute either specifies a user defined behaviour or
        one of the OTP standard behaviours \texttt{gen\_server},
        \texttt{gen\_fsm}, \texttt{gen\_event} or
        \texttt{supervisor}. The spelling ``behavior'' is also accepted.

\end{itemize}


\subsection{Macros y definici\'on de grabaciones}

Tanto las grabaciones y las macros se graban de la misma forma que se graban los atributos de los m\'odulos: 

\begin{erlang}
-record(Record,Fields).

-define(Macro,Replacement).
\end{erlang}
Las grabaciones y los macros tambi\'en se pueden realizar entre funciones, para ello el \'unico requisito es que la definici\'on vaya antes que su primer uso. (Sobre los registros consulta la secci\'on \ref{datatypes:record} and about macros see chapter \ref{macros}.)

\subsection{File inclusion}

File inclusion is specified in the same way as module attributes:

\begin{erlang}
-include(File).

-include_lib(File).
\end{erlang}

\texttt{File} is a string that represents a file name. Include files
are typically used for record and macro definitions that are shared by
several modules. By convention, the extension \texttt{.hrl} is used
for include files.

\begin{erlang}
-include("my_records.hrl").
-include("incdir/my_records.hrl").
-include("/home/user/proj/my_records.hrl").
\end{erlang}

If \texttt{File} starts with a path component \texttt{\$Var}, then the value of
the environment variable \texttt{Var} (returned by
\texttt{os:getenv(Var)}) is substituted for \texttt{\$Var}.

\begin{erlang}
-include("$PROJ_ROOT/my_records.hrl").
\end{erlang}
%%$ texmaker parser bug

\texttt{include\_lib} is similar to \texttt{include}, but the
first path component is assumed to be the name of an application.

\begin{erlang}
-include_lib("kernel/include/file.hrl").
\end{erlang}

The code server uses \texttt{code:lib\_dir(kernel)} to find the
directory of the current (latest) version of \texttt{kernel}, and then
the subdirectory \texttt{include} is searched for the file \texttt{file.hrl}.


\section{Comments}
Comments may appear anywhere in a module except within strings and
quoted atoms.  A comment begins with the percentage character
(\texttt{\%}) and covers the rest of the line but not the
end-of-line. The terminating end-of-line has the effect of a blank.


\section{Character Set}
Erlang handles the full Latin-1 (ISO-8859-1) character set. Thus all
Latin-1 printable characters can be used and displayed without the
escape backslash. Atoms and variables can use all Latin-1 characters.

\vspace*{12pt}
\begin{center}
\begin{tabular}{|>{\raggedright}p{52pt}|>{\raggedright}p{53pt}|>{\raggedright}p{103pt}|>{\raggedright}p{87pt}|}
\hline
\multicolumn{4}{|p{297pt}|}{Character classes}\tabularnewline
\hline
Octal & Decimal~ &   & Class\tabularnewline
\hline
40 -  57 & 32 - 47 &  ! \texttt{"} \# \$ \% \& ' / & Punctuation
characters\tabularnewline
\hline
60 -  71 & 48 - 57 & 0 - 9 & Decimal digits\tabularnewline
\hline
72 - 100 & 58 - 64 & : ; \texttt{<} = \texttt{>} @ & Punctuation characters\tabularnewline
\hline
101 - 132 &  65 - 90 & A - Z & Uppercase letters\tabularnewline
\hline
133 - 140 &  91 - 96 & [ \textbackslash{} ] \textasciicircum{} \_ ` & Punctuation
characters\tabularnewline
\hline
141 - 172 &  97 - 122 & a  -  z & Lowercase letters\tabularnewline
\hline
173 - 176 & 123 - 126 & \{ \textbar{} \} \textasciitilde{} & Punctuation characters\tabularnewline
\hline
200 - 237 & 128 - 159 ~ &   & Control characters \tabularnewline
\hline
240 - 277 & 160 - 191 & - ¿  & Punctuation characters \tabularnewline
\hline
300 - 326 & 192 - 214 & À - Ö  & Uppercase letters \tabularnewline
\hline
327  & 215 & ×  & Punctuation character \tabularnewline
\hline
330 - 336 & 216 - 222 & Ø - Þ  & Uppercase letters \tabularnewline
\hline
337 - 366 & 223 - 246 & ß - ö  & Lowercase letters \tabularnewline
\hline
367  & 247 & ÷  & Punctuation character \tabularnewline
\hline
370 - 377 & 248 - 255 & ø - ÿ  & Lowercase letters \tabularnewline
\hline
\end{tabular}
\end{center}

% because of where this lands, force a page break to avoid orphan.
\newpage
\section{Reserved words}

%\vspace{12pt}

The following are reserved words in Erlang:

\begin{erlang}
after and andalso band begin bnot bor bsl bsr bxor case catch cond
div end fun if let not of or orelse receive rem try when xor
\end{erlang}
