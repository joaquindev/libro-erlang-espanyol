\chapter{Estructura de un programa en Erlang}

\section{Sintaxis de los m\'odulos}

Un programa de Erlang est\'a compuesto \textbf{por m\'odulos} donde cada m\'odulo
es un fichero de texto con extensi\'on \textbf{.erl}. En programas peque\~nos, todos
los m\'odulos residen en el mismo directorio. Un m\'odulo consiste de atributos y 
definici\'on de funciones.

\begin{erlang}
-module(demo).
-export([double/1]).

double(X) -> times(X, 2).

times(X, N) -> X * N.
\end{erlang}

El m\'odulo \texttt{demo} de arriba consiste en la funci\'on \texttt{times/2}
y es local al m\'odulo y de la funci\'on \texttt{double/1} que se exporta
y que puede ser llamada desde fuera del m\'odulo.

\texttt{demo:double(10)} \resultingin \texttt{20}\hfill
(la flecha $\Rightarrow$ se lee como ``da como resultado'') \texttt{double/1} significa la funci\'on ``double'' con \textit{one}(un)
argumento. Una funci\'on \texttt{double/2} que recibe \textit{two}(dos) argumentos se considera una funci\'on diferente. El n\'umero de argumentos se llama la \textbf{aridad} de la funci\'on.

\section{Atributos de m\'odulo}
Un \textbf{atributo de un m\'odulo} define una propiedad concreta de un m\'odulo y consiste en una \textbf{etiqueta}(tag) y un \textbf{valor}(value):
\texttt{-Tag(valor).} 

\texttt{Tag} tiene que se un \'atomo, mientras que \texttt{valor} tiene que ser un t\'ermino literal (consulta el cap\'itulo \ref{datatypes}). Cualquier atributo de un m\'odulo puede ser especificado. Los atributos se guardan en el c\'odigo compilado y puede ser recuperado llamando a la funci\'on \texttt{Module:module\_info(attributes).}

\subsection{Atributos de m\'odulo predefinidos}
Los atributos de m\'odulo predefinidos tienen que ponerse antes de cualquier declaraci\'on de funci\'on.

\begin{itemize}
	\item \begin{erlangim}
	-module(Module).
	\end{erlangim}
	Este atributo es obligatorio y debe ser especificado primero. Define el nombre del m\'odulo. El nombre del m\'odulo, un \'atomo (consultar secci\'on \ref{datatypes:atom}), debe ser el mismo que el nombre del fichero sin la extensi\'on `\texttt{.erl}'.

	\item \begin{erlangim}
	-export([Func1/Arity1, ..., FuncN/ArityN]).
	\end{erlangim}
	Este atributo especifica qu\'e en el m\'odulo pueden ser llamadas desde fuera del m\'odulo. Cada nombre de funci\'on \texttt{FuncX} es un \'atomo y \texttt{ArityX} es un valor entero.

	\item \begin{erlangim}
	-import(Module,[Func1/Arity1, ..., FuncN/ArityN]).
	\end{erlangim}
    Este atributo indica un \texttt{Module} del cual se importar\'a una lista de funciones, por ejemplo:  
    \begin{erlangim}
	-import(demo, [double/1]).
	\end{erlangim}
    Esto signfica que es posible escribir \texttt{double(10)} en lugar de la versi\'on m\'as larga \texttt{demo:double(10)} la cual puede resultar poco pr\'actica si la funci\'on se utiliza de manera frecuente.

	\item \begin{erlangim}
	-compile(Options).
	\end{erlangim}
	Opciones del compilador.

	\item \begin{erlangim}
	-vsn(Vsn).
	\end{erlangim}
    La versi\'on del m\'odulo. Si este atributo no se especifica, la versi\'on ser\'a por defecto el valor del \texttt{checksum} del m\'odulo.

	\item \begin{erlangim}
	-behaviour(Behaviour).
	\end{erlangim}
    Este atributo especifica puede especificar dos cosas: el comportamiento definido por un usuario o un comportamiento est\'andar disponible en OTP \texttt{gen\_server}, \texttt{gen\_fsm}, \texttt{gen\_event} or
        \texttt{supervisor}. Escribir la palabra ``behavior'' es una opci\'on que tambi\'en es aceptada.

\end{itemize}


\subsection{Macros y definici\'on de estructuras de datos}

Tanto las grabaciones y las estructuras de datos se graban de la misma forma que se graban los atributos de los m\'odulos: 

\begin{erlang}
-record(Record,Fields).

-define(Macro,Replacement).
\end{erlang}
Las grabaciones y los estructuras de datos tambi\'en se pueden realizar entre funciones, para ello el \'unico requisito es que la definici\'on vaya antes que su primer uso. (Sobre los registros consulta la secci\'on \ref{datatypes:record} y sobre las estructuras de datos consulta el cap\'itulo \ref{estructuras de datos}.)

\subsection{Incluyendo un fichero}

La inclusi\'on de un fichero se especifica de la misma manera que los atributos del m\'odulo: 

\begin{erlang}
-include(File).

-include_lib(File).
\end{erlang}

\texttt{File} es una cadena de texto que representa el nombre de un fichero. Los ficheros \texttt{Include
son utilizados normalmente por las definiciones de los registros y los macros compartidos por diferentes 
m\'odulos. Por convenci\'on, la extensi\'on \texttt{.hrl} se utilizada para incluir ficheros.

\begin{erlang}
-include("my_records.hrl").
-include("incdir/my_records.hrl").
-include("/home/user/proj/my_records.hrl").
\end{erlang}

Si el \texttt{File} empieza por una ruta \texttt{\$Var}, entonces el valor de la variable de entorno \texttt{Var} (devuelta 
por \texttt{os:getenv(Var)}) se sustituye por \texttt{\$Var}.

\begin{erlang}
-include("$PROJ_ROOT/my_records.hrl").
\end{erlang}
%%$ texmaker parser bug

\texttt{include\_lib} es similar a \texttt{include}, pero el primer elemento de la ruta se asume que es el nombre de la aplicaci\'on.

\begin{erlang}
-include_lib("kernel/include/file.hrl").
\end{erlang}
El c\'odigo del servidor utiliza \texttt{code:lib\_dir(kernel)} para encontrar el directorio
de la \'ultima y actualizada versi\'on del \texttt{kernel}, y entonces el subdirectorio
\texttt{include} es donde se buscar\'a para encontrar el fichero \texttt{file.hrl}.

\section{Comentarios}
Los comentarios pueden estar en cualquier m\'odulo excepto dentro de cadenas de texto 
y \'atomos en \texttt{quoted}. Un comentario empieza con el car\'acter 
(\texttt{\%}) y llega hasta el final de la l\'inea pero no incluye el salto de l\'inea. El efecto que
tiene el terminar una l\'inea con un salto de l\'inea es el efecto de cortar el comentario.


\section{Conjunto de caracteres}
Erlang funciona con el conjunto de caracteres Latin-1 (ISO-8859-1). Es decir, todos los caracteres que existen en dicho conjunto podr\'an ser utilizados sin necesidad de utilizar el car\'acter de escape. Por tanto, los \'atomos y las variables podr\'an ser nombradas haciendo uso del conjunto de caracteres Latin-1. 

\vspace*{12pt}
\begin{center}
\begin{tabular}{|>{\raggedright}p{52pt}|>{\raggedright}p{53pt}|>{\raggedright}p{103pt}|>{\raggedright}p{87pt}|}
\hline
\multicolumn{4}{|p{297pt}|}{Tipos de caracteres}\tabularnewline
\hline
Octal & Decimal~ &   & Clase\tabularnewline
\hline
40 -  57 & 32 - 47 &  ! \texttt{"} \# \$ \% \& ' / & Signos de puntuaci\'on\tabularnewline
\hline
60 -  71 & 48 - 57 & 0 - 9 & D\'igitos decimales\tabularnewline
\hline
72 - 100 & 58 - 64 & : ; \texttt{<} = \texttt{>} @ & Signos de puntuaci\'on\tabularnewline
\hline
101 - 132 &  65 - 90 & A - Z & Letras en may\'usculas\tabularnewline
\hline
133 - 140 &  91 - 96 & [ \textbackslash{} ] \textasciicircum{} \_ ` & Signos de puntuaci\'on\tabularnewline
\hline
141 - 172 &  97 - 122 & a  -  z & Letras en min\'usculas\tabularnewline
\hline
173 - 176 & 123 - 126 & \{ \textbar{} \} \textasciitilde{} & Signos de puntuaci\'on\tabularnewline
\hline
200 - 237 & 128 - 159 ~ &   & Caracteres de control\tabularnewline
\hline
240 - 277 & 160 - 191 & - ¿  & Signos de puntuaci\'on\tabularnewline
\hline
300 - 326 & 192 - 214 & À - Ö  & Letras en may\'usculas\tabularnewline
\hline
327  & 215 & ×  & Signos de puntuaci\'on\tabularnewline
\hline
330 - 336 & 216 - 222 & Ø - Þ  & Letras may\'usculas \tabularnewline
\hline
337 - 366 & 223 - 246 & ß - ö  & Letras min\'usculas\tabularnewline
\hline
367  & 247 & ÷  & Signos de puntuaci\'on\tabularnewline
\hline
370 - 377 & 248 - 255 & ø - ÿ  & Letars min\'usculas\tabularnewline
\hline
\end{tabular}
\end{center}

% because of where this lands, force a page break to avoid orphan.
\newpage
\section{Palabras reservadas}

%\vspace{12pt}

Las siguientes palabras se consideran palabras reservadas en Erlang: 

\begin{erlang}
after and andalso band begin bnot bor bsl bsr bxor case catch cond
div end fun if let not of or orelse receive rem try when xor
\end{erlang}
