\chapter{Estructura de un programa en Erlang}

\section{Sintaxis de los m\'odulos}

Un programa de Erlang est\'a compuesto \textbf{por m\'odulos} donde cada m\'odulo
es un fichero de texto con extensi\'on \textbf{.erl}. En programas peque\~nos, todos
los m\'odulos residen en el mismo directorio. Un m\'odulo consiste de atributos y 
definici\'on de funciones.

\begin{erlang}
-module(demo).
-export([double/1]).

double(X) -> times(X, 2).

times(X, N) -> X * N.
\end{erlang}

El m\'odulo \texttt{demo} de arriba consiste en la funci\'on \texttt{times/2}
y es local al m\'odulo y de la funci\'on \texttt{double/1} que se exporta
y que puede ser llamada desde fuera del m\'odulo.

\texttt{demo:double(10)} \resultingin \texttt{20}\hfill
(la flecha $\Rightarrow$ se lee como ``da como resultado'') \texttt{double/1} significa la funci\'on ``double'' con \textit{one}(un)
argumento. Una funci\'on \texttt{double/2} que recibe \textit{two}(dos) argumentos se considera una funci\'on diferente. El n\'umero de argumentos se llama la \textbf{aridad} de la funci\'on.

\section{Atributos de m\'odulo}
Un \textbf{atributo de un m\'odulo} define una propiedad concreta de un m\'odulo y consiste en una \textbf{etiqueta}(tag) y un \textbf{valor}(value):
\texttt{-Tag(valor).} 

\texttt{Tag} tiene que se un \'atomo, mientras que \texttt{valor} tiene que ser un t\'ermino literal (consulta el cap\'itulo \ref{datatypes}). Cualquier atributo de un m\'odulo puede ser especificado. Los atributos se guardan en el c\'odigo compilado y puede ser recuperado llamando a la funci\'on \texttt{Module:module\_info(attributes).}

\subsection{Atributos de m\'odulo predefinidos}
Los atributos de m\'odulo predefinidos tienen que ponerse antes de cualquier declaraci\'on de funci\'on.

\begin{itemize}
	\item \begin{erlangim}
	-module(Module).
	\end{erlangim}
	Este atributo es obligatorio y debe ser especificado primero. Define el nombre del m\'odulo. El nombre del m\'odulo, un \'atomo (consultar secci\'on \ref{datatypes:atom}), debe ser el mismo que el nombre del fichero sin la extensi\'on `\texttt{.erl}'.

	\item \begin{erlangim}
	-export([Func1/Arity1, ..., FuncN/ArityN]).
	\end{erlangim}
	Este atributo especifica qu\'e en el m\'odulo pueden ser llamadas desde fuera del m\'odulo. Cada nombre de funci\'on \texttt{FuncX} es un \'atomo y \texttt{ArityX} es un valor entero.

	\item \begin{erlangim}
	-import(Module,[Func1/Arity1, ..., FuncN/ArityN]).
	\end{erlangim}
    Este atributo indica un \texttt{Module} del cual se importar\'a una lista de funciones, por ejemplo:  
    \begin{erlangim}
	-import(demo, [double/1]).
	\end{erlangim}
    Esto signfica que es posible escribir \texttt{double(10)} en lugar de la versi\'on m\'as larga \texttt{demo:double(10)} la cual puede resultar poco pr\'actica si la funci\'on se utiliza de manera frecuente.

	\item \begin{erlangim}
	-compile(Options).
	\end{erlangim}
	Opciones del compilador.

	\item \begin{erlangim}
	-vsn(Vsn).
	\end{erlangim}
    La versi\'on del m\'odulo. Si este atributo no se especifica, la versi\'on ser\'a por defecto el valor del \texttt{checksum} del m\'odulo.

	\item \begin{erlangim}
	-behaviour(Behaviour).
	\end{erlangim}
    Este atributo especifica puede especificar dos cosas: el comportamiento definido por un usuario o un comportamiento est\'andar disponible en OTP \texttt{gen\_server}, \texttt{gen\_fsm}, \texttt{gen\_event} or
        \texttt{supervisor}. Escribir la palabra ``behavior'' es una opci\'on que tambi\'en es aceptada.

\end{itemize}


\subsection{Macros y definici\'on de estructuras de datos}

Tanto las grabaciones y las estructuras de datos se graban de la misma forma que se graban los atributos de los m\'odulos: 

\begin{erlang}
-record(Record,Fields).

-define(Macro,Replacement).
\end{erlang}
Las grabaciones y los estructuras de datos tambi\'en se pueden realizar entre funciones, para ello el \'unico requisito es que la definici\'on vaya antes que su primer uso. (Sobre los registros consulta la secci\'on \ref{datatypes:record} y sobre las estructuras de datos consulta el cap\'itulo \ref{estructuras de datos}.)

\subsection{Incluyendo un fichero}

La inclusi\'on de un fichero se especifica de la misma manera que los atributos del m\'odulo: 

\begin{erlang}
-include(File).

-include_lib(File).
\end{erlang}

\texttt{File} es una cadena de texto que representa el nombre de un fichero. Los ficheros \texttt{Include
son utilizados normalmente por las definiciones de los registros y los macros compartidos por diferentes 
m\'odulos. Por convenci\'on, la extensi\'on \texttt{.hrl} se utilizada para incluir ficheros.

\begin{erlang}
-include("my_records.hrl").
-include("incdir/my_records.hrl").
-include("/home/user/proj/my_records.hrl").
\end{erlang}

Si el \texttt{File} empieza por una ruta \texttt{\$Var}, entonces el valor de la variable de entorno \texttt{Var} (devuelta 
por \texttt{os:getenv(Var)}) se sustituye por \texttt{\$Var}.

\begin{erlang}
-include("$PROJ_ROOT/my_records.hrl").
\end{erlang}
%%$ texmaker parser bug

\texttt{include\_lib} es similar a \texttt{include}, pero el primer elemento de la ruta 
 se asume que es el nombre de la aplicaci\'on.

\begin{erlang}
-include_lib("kernel/include/file.hrl").
\end{erlang}
El c\'odigo del servidor utiliza 

The code server uses \texttt{code:lib\_dir(kernel)} to find the
directory of the current (latest) version of \texttt{kernel}, and then
the subdirectory \texttt{include} is searched for the file \texttt{file.hrl}.


\section{Comments}
Comments may appear anywhere in a module except within strings and
quoted atoms.  A comment begins with the percentage character
(\texttt{\%}) and covers the rest of the line but not the
end-of-line. The terminating end-of-line has the effect of a blank.


\section{Character Set}
Erlang handles the full Latin-1 (ISO-8859-1) character set. Thus all
Latin-1 printable characters can be used and displayed without the
escape backslash. Atoms and variables can use all Latin-1 characters.

\vspace*{12pt}
\begin{center}
\begin{tabular}{|>{\raggedright}p{52pt}|>{\raggedright}p{53pt}|>{\raggedright}p{103pt}|>{\raggedright}p{87pt}|}
\hline
\multicolumn{4}{|p{297pt}|}{Character classes}\tabularnewline
\hline
Octal & Decimal~ &   & Class\tabularnewline
\hline
40 -  57 & 32 - 47 &  ! \texttt{"} \# \$ \% \& ' / & Punctuation
characters\tabularnewline
\hline
60 -  71 & 48 - 57 & 0 - 9 & Decimal digits\tabularnewline
\hline
72 - 100 & 58 - 64 & : ; \texttt{<} = \texttt{>} @ & Punctuation characters\tabularnewline
\hline
101 - 132 &  65 - 90 & A - Z & Uppercase letters\tabularnewline
\hline
133 - 140 &  91 - 96 & [ \textbackslash{} ] \textasciicircum{} \_ ` & Punctuation
characters\tabularnewline
\hline
141 - 172 &  97 - 122 & a  -  z & Lowercase letters\tabularnewline
\hline
173 - 176 & 123 - 126 & \{ \textbar{} \} \textasciitilde{} & Punctuation characters\tabularnewline
\hline
200 - 237 & 128 - 159 ~ &   & Control characters \tabularnewline
\hline
240 - 277 & 160 - 191 & - ¿  & Punctuation characters \tabularnewline
\hline
300 - 326 & 192 - 214 & À - Ö  & Uppercase letters \tabularnewline
\hline
327  & 215 & ×  & Punctuation character \tabularnewline
\hline
330 - 336 & 216 - 222 & Ø - Þ  & Uppercase letters \tabularnewline
\hline
337 - 366 & 223 - 246 & ß - ö  & Lowercase letters \tabularnewline
\hline
367  & 247 & ÷  & Punctuation character \tabularnewline
\hline
370 - 377 & 248 - 255 & ø - ÿ  & Lowercase letters \tabularnewline
\hline
\end{tabular}
\end{center}

% because of where this lands, force a page break to avoid orphan.
\newpage
\section{Reserved words}

%\vspace{12pt}

The following are reserved words in Erlang:

\begin{erlang}
after and andalso band begin bnot bor bsl bsr bxor case catch cond
div end fun if let not of or orelse receive rem try when xor
\end{erlang}
