\chapter{Tipos de datos (terms)}
\label{datatypes}

\section{Tipos de datos Unarios}

\subsection{\'Atomos}
\label{datatypes:atom}
Un \textbf{\'atomo} es un nombre simb\'olico, tambi\'en conocido como \textit{literal}. Los \'atomos comienzan con una letra en min\'usculas y pueden contener caracteres alfanum\'ericos, subrayados (\texttt{\_}) u el s\'imbolo arroba (\texttt{@}).
Alternativamente, los \'atomos pueden ser especificados utilizando las comillas simples (\texttt{'}), las cuales son necesarias cuando el \'atomo empieza con un car\'acter en may\'usculas o contienen otros caracteres adem\'as del car\'acter subrayado o arroba.

\texttt{hello}

\texttt{phone\_number}

\texttt{'Monday'}

\texttt{'phone number'}

\texttt{'Cualquier cosa dentro de comillas \textbackslash n\textbackslash 012'}
\hfill(consultar secci\'on \ref{datatypes:escapeseq})


\subsection{Booleano}
\label{datatypes:boolean}
En Erlang no existe el tipo \textbf{booleano}. Los \'atomos \texttt{true} y \texttt{false} son los que los reemplazan.

\texttt{2 =< 3} \resultingin \texttt{true} \\
\texttt{true or false} \resultingin \texttt{true}

\subsection{Enteros}
\label{datatypes:integer}
Adem\'as de la forma normal de escribir \textbf{integers} Erlang
ofrece otras notaciones. \texttt{\$Char} es el valor num\'erico del car\'acter
`\texttt{Char}' (que puede ser una secuencia de escape) y \texttt{Base\#Value} es un entero en base \texttt{Base}, el cual debe ser un valor entero dentro del rango $2..36$.

\texttt{42} \resultingin \texttt{42} \\
\$A  \resultingin \texttt{65} \\
\texttt{\$\textbackslash n} \resultingin \texttt{10}\hfill(see section \ref{datatypes:escapeseq}) \\
\texttt{2\#101} \resultingin \texttt{5} \\
\texttt{16\#1f} \resultingin \texttt{31}


\subsection{Valores en punto flotante}
\label{datatypes:float}
Un valor en punto flotante es un valor real escrito \texttt{Num}[\texttt{eExp}]
donde \texttt{Num} es un n\'umero decimal entre 0.01 y 10000, y \texttt{Exp} (opcional) es un valor entero con signo especificando el exponente. Por ejemplo:

\texttt{2.3e-3} \resultingin \texttt{2.30000e-3}\hfill
(corresponde a 2.3*10\textsuperscript{-3})


\subsection{Referencias}
\label{datatypes:reference}
Una \textbf{reference} es un t\'ermino \'unico dentro de un sistema en tiempo de ejecuci\'on Erlang creado mediante la funci\'on propia \texttt{make\_ref/0}. (Para m\'as informaci\'on sobre \'esta funci\'on o sobre \textit{BIF}s, consulta la secci\'on \ref{functions:bifs}.)


\subsection{Puertos}
\label{datatypes:port}
Un \textbf{identificador de un puerto} identifica un puerto (consultar el cap\'itulo  \ref{ports}).


\subsection{Pids, identificadores de proceso}
\label{datatypes:pid}
Un \textbf{identificador de proceso}, \textit{pid}, identifica un proceso de forma \'unica (consulta el cap\'itulo \ref{processes}).


\subsection{Funs}
\label{datatypes:fun}
Una \textit{fun} sirve para identificar a un \textbf{objeto funcional} (consultar secci\'on \ref{functions:funs}).

\section{Tipos de datos compuestos}

\subsection{Tuplas}
\label{datatypes:tuple}
Una \textbf{tupla} es un tipo de dato compuesto que contiene un \textbf{n\'umero fijo de t\'erminos} encerrados entre unos corchetes.

\texttt{\{Term1,...,TermN\}}

Cada \texttt{TermX} en la tupla se llama \textbf{elemento}. El n\'umero de elementos se concoe como el \textbf{tamanyo}de la tupla.

\begin{center}
\begin{tabular}{|>{\raggedright}p{134pt}|>{\raggedright}p{186pt}|}
\hline
\multicolumn{2}{|p{321pt}|}{BIFs para manipular funciones}\tabularnewline
\hline
\texttt{size(Tuple)} & Returns the size of \texttt{Tuple}\tabularnewline
\hline
\texttt{element(N,Tuple)} & Returns the \texttt{N}\textsuperscript{th} element in \texttt{Tuple}\tabularnewline
\hline
\texttt{setelement(N,Tuple,Expr)} & Returns a new tuple copied from \texttt{Tuple} except that the
\texttt{N}\textsuperscript{th} element is replaced by \texttt{Expr}\tabularnewline
\hline
\end{tabular}
\end{center}

\texttt{P = \{adam, 24, \{july, 29\}\}} \resultingin \texttt{P} is bound to \texttt{\{adam, 24, \{july, 29\}\}} \\
\texttt{element(1, P)} \resultingin \texttt{adam} \\
\texttt{element(3, P)} \resultingin \texttt{ \{july,29\}} \\
\texttt{P2 = setelement(2, P, 25)} \resultingin \texttt{P2} is bound to \texttt{\{adam, 25, \{july, 29\}\}} \\
\texttt{size(P)} \resultingin \texttt{3} \\
\texttt{size(\{\})} \resultingin \texttt{0} \\


\subsection{Registros}
\label{datatypes:record}
Un \textbf{registro} es \textit{una tupla con nombre} con elementos llamados \textbf{campos}. El tipo de un registro se define como un atributo del m\'odulo. Por ejemplo:

\begin{erlang}
-record(Rec, {Field1 [= Value1],
              ...
              FieldN [= ValueN]}).
\end{erlang}

\texttt{Rec} and \texttt{Fields} son \'atomos y a cada \texttt{FieldX}
se le puede dar un valor opcional por defecto \texttt{ValueX}. Esta definici\'on puede ponerse
entre las funciones de un m\'odulo, pero s\'olamente antes de que sea utilizada. Si un objeto de tipo 
registro es utilizado por varios m\'odulos, se recomienda ponerlo en un fichero diferente y que se proceda
a su inclusi\'on.

Para crear un nuevo registro de tipo \texttt{Rec} utilizaremos una expresi\'on as\'i:

\begin{erlang}
#Rec{Field1=Expr1, ..., FieldK=ExprK [, _=ExprL]}
\end{erlang}
Los campos no tienen por qué estar en el mismo orden en el que están en la definición 
del registro. Los campos que se omitan van a tener el valor por defecto. Si se utiliza
la cláusula final se utiliza, los campos que se han omitido van a tener el valor \texttt{ExprL}.

Los campos que no tienen valor y que además son omitidos tendrán el valor *undefined*. 

Para recuperar el valor de un registro utilizaremos la expresión: 
``\texttt{Variable\#Rec.Field}''.

\begin{erlang}
-module(employee).
-export([new/2]).
-record(person, {name, age, employed=erixon}).

new(Name, Age) -> #person{name=Name, age=Age}.
\end{erlang}

La función \texttt{employee:new/2} puede ser utilizada desde otro módulo, la cual deberá
de incluir la misma definición del registro de \texttt{person}.

\texttt{\{P = employee:new(ernie,44)\}} \resultingin \texttt{\{person, ernie, 44,
erixon\}} \\
\texttt{P\#person.age} \resultingin \texttt{44} \\
\texttt{P\#person.employed} \resultingin \texttt{erixon}

Al trabajar con registros desde el *shell* de Erlang, podremos utilizar las funciones \texttt{rd(RecordName, RecordDefinition)} y \texttt{rr(Module)}
para definir las definiciones de los registros. Para más información véase el \textit{Manual de Referencia de Erlang}.


\subsection{Listas}
\label{datatypes:list}
Una \textbf{lista} es un tipo de datos compuesto que va a tener un número \textit{variable}
de \textbf{términos} encerrados dentro de unos corchetes.

\texttt{[Term1,...,TermN]}

Cada t\'ermino \texttt{TermX} en la lista se llama \textbf{elemento}. La \textbf{longitud} de una lista hace referencia al n\'umero de elementos. Com\'unmente en programaci\'on funcional, al primer elemento se le conoce como la cabeza o \textbf{head} de la lista y el resto de la lista (desde la 2\textsuperscript{nd} elemento en adelante) se le conoce como la cola \textbf{tail} de la lista. Date cuenta que los elementos individuales dentro de una lista no tienen que ser obligatoriamente del mismo tipo, aunque lo m\'as com\'un es que s\'i que lo sean. Aunque es una buena pr\'actica y recomendable que se mantengan en la medida de lo posible del mismo tiepo ya que en el caso de mezclar los tipos, es m\'as com\'un utilizar registros \textbf{records}.

\begin{center}
\begin{tabular}{|>{\raggedright}p{90pt}|>{\raggedright}p{230pt}|}
\hline
\multicolumn{2}{|p{321pt}|}{BIFs to manipulate lists}\tabularnewline
\hline
\texttt{length(List)} & Returns the length of \texttt{List}\tabularnewline
\hline
\texttt{hd(List)} & Returns the 1\textsuperscript{st} (head) element of \texttt{List}\tabularnewline
\hline
\texttt{tl(List)} & Returns \texttt{List} with the 1\textsuperscript{st} element removed (tail)\tabularnewline
\hline
\end{tabular}
\end{center}

El operador barra vertical (\textbar{}) separa a los elementos que est\'an en la cabeza del resto de elementos. Por ejemplo: 

\texttt{[H | T]  = [1, 2, 3, 4, 5]} \resultingin \texttt{H=1} and \texttt{T=[2, 3, 4, 5]} \\
\texttt{[X, Y | Z] = [a, b, c, d, e]} \resultingin \texttt{X=a}, \texttt{Y=b} and \texttt{Z=[c, d, e]}

Impl\'icitamente una lista va a terminar siempre con una lista vac\'ia, p. e.~\texttt{[a, b]} es lo mismo
que \texttt{[a, b | []]}.  Y tambi\'en, una lista \texttt{[a, b | c]}
es y se considera una lista \textbf{mal formada} y su uso deber\'ia ser evitado (ya que el \'atomo '\texttt{c}' \textit{no} es una lista).
Las listas se prestan por naturaleza a la programación funcional recursiva. Por ejemplo la siguiente función `\texttt{sum}' calcula la suma de una lista, y `\texttt{double}' multiplica cada elemento de la lista por 2, construyendo y devolviendo una nueva lista a medida que va recorriendo la lista. 

\begin{erlang}
sum([]) -> 0;
sum([H | T]) -> H + sum(T).

double([]) -> [];
double([H | T]) -> [H*2 | double(T)].
\end{erlang}

Las definiciones de arriba introducen el uso del \textit{pattern matching} (reconocimiento de patrones), descrito en el capítulo \ref{patterns}. Los patrones de este estilo son típicos en programación recursiva de forma que proporcionan explícitamente un caso base (``base case'') (para una lista vacía en este caso). 

Para trabajar con listas, el operador \texttt{++} une dos listas juntas (de forma que incluye los elementos pasados como segundo argumento al primero) y además devuelve la lista resultante. El operador \texttt{--} produce una lista que es una copia de su primer argumento, excepto que para cada elemento en el segundo argumento, la primera ocurrencia de este elemento (si es que lo hay), será eliminada. 

\texttt{[1,2,3] ++ [4,5]} \resultingin \texttt{[1,2,3,4,5]}
\texttt{[1,2,3,2,1,2] -- [2,1,2]} \resultingin \texttt{[3,1,2]}

Una colección de funciones para procesar listas se puede encontrar en \texttt{STDLIB} en el módulo \texttt{lists} (listas). 


\subsection{Cadenas de caracteres}
\label{datatypes:string}
Las \textbf{cadenas de caracteres} son caracteres delimitados por comillas dobles que realmente se guardan como una lista de caracteres. 

\texttt{"abcdefghi"} es lo mismo que \texttt{[97,98,99,100,101,102,103,104,105]}

\texttt{""} es lo mismo que \texttt{[]}

Dos cadenas adyacentes se concatenarán en una en tiempo de compilación y por tanto no aumentarán ni afectarán el tiempo de ejecución. 

\texttt{"string" "42"} \resultingin \texttt{"string42"}

\subsection{Binarios}
\label{datatypes:binary}
Un dato binario es un trozo de memoria (por defecto son bytes de 8-bits) y que no tiene ningún tipo asingado. 

\texttt{<}\texttt{<Elem1,...,ElemN>}\texttt{>}

Cada \texttt{ElemX} se especifica como \texttt{Value (valor)}[\texttt{:Size (dimensión)}][\texttt{/TypeSpecifierList}].

\begin{center}
\begin{tabular}{|>{\raggedright}p{73pt}|>{\raggedright}p{81pt}|>{\raggedright}p{147pt}|}
\hline
\multicolumn{3}{|p{297pt}|}{Especificación del elemento}\tabularnewline
\hline
\texttt{Value} & \texttt{Size} & \texttt{TypeSpecifierList}\tabularnewline
\hline
Debe ser evaluado a un entero, coma flotante o binario & Debe evaluarse como entero & Una secuencia opcional de especificadores de tipo, en cualquier orden, separada por guiones (-)\tabularnewline
\hline
\end{tabular}
\end{center}

% page break at this point, so turned into two chunks.
\begin{center}
\begin{tabular}{|>{\raggedright}p{47pt}|>{\raggedright}p{115pt}|>{\raggedright}p{147pt}|}
\hline
\multicolumn{3}{|p{297pt}|}{Type specifiers}\tabularnewline
\hline
Type & \texttt{integer} \textbar{} \texttt{float} \textbar{} \texttt{binary} & Default
is \texttt{integer}\tabularnewline
\hline
Signedness & \texttt{signed} \textbar{} \texttt{unsigned} & Default is
\texttt{unsigned}\tabularnewline
\hline
Endianness & \texttt{big} \textbar{} \texttt{little} \textbar{} \texttt{native} & CPU
dependent. Default is \texttt{big}\tabularnewline
\hline
Unit & \texttt{unit:}\textit{IntegerLiteral} & Allowed range is $1..256$.
Default is 1 for integer and float, and 8 for binary\tabularnewline
\hline
\end{tabular}
\end{center}

El valor de \texttt{Size} multiplicado por las unidades da el número de bits para el segmento. Cada segmento puede consistir en cero o más bits pero el número total de bits debe ser un múltiplo de 8, o se producirá un error en tiempo de ejecución del tipo \texttt{badarg}. Además, un segmento de tipo binario debe tener una dimensión que sea divisible por ocho (uniformemente). 

Los binarios no pueden anidarse.

\begin{erlang}
<<1, 17, 42>>       % <<1, 17, 42>>
<<"abc">>           % <<97, 98, 99>> (The same as <<$a, $b, $c>>)
<<1, 17, 42:16>>    % <<1,17,0,42>>
<<>>                % <<>>
<<15:8/unit:10>>    % <<0,0,0,0,0,0,0,0,0,15>>
<<(-1)/unsigned>>   % <<255>>
\end{erlang}


\section{Escape sequences}
\label{datatypes:escapeseq}
Escape sequences are allowed in strings and quoted atoms.

\begin{center}
\begin{tabular}{|>{\raggedright}p{91pt}|>{\raggedright}p{229pt}|}
\hline
\multicolumn{2}{|p{321pt}|}{Escape sequences}\tabularnewline
\hline
\textbackslash{}b & Backspace\tabularnewline
\hline
\textbackslash{}d & Delete\tabularnewline
\hline
\textbackslash{}e & Escape\tabularnewline
\hline
\textbackslash{}f & Form feed\tabularnewline
\hline
\textbackslash{}n & New line\tabularnewline
\hline
\textbackslash{}r & Carriage return\tabularnewline
\hline
\textbackslash{}s & Space\tabularnewline
\hline
\textbackslash{}t & Tab\tabularnewline
\hline
\textbackslash{}v & Vertical tab\tabularnewline
\hline
\textbackslash{}XYZ, \textbackslash{}XY, \textbackslash{}X & Character with octal
representation XYZ, XY or X\tabularnewline
\hline
\textbackslash{}\textasciicircum{}A .. \textbackslash{}\textasciicircum{}Z & Control
A to control Z\tabularnewline
\hline
\textbackslash{}\textasciicircum{}a .. \textbackslash{}\textasciicircum{}z & Control
A to control Z\tabularnewline
\hline
\textbackslash{}' & Single quote\tabularnewline
\hline
\textbackslash{}\textbf{\texttt{"}} & Double quote\tabularnewline
\hline
\textbackslash{}\textbackslash{} & Backslash\tabularnewline
\hline
\end{tabular}
\end{center}

\section{Type conversions}
There are a number of built-in functions for type conversion:

\begin{center}
\begin{tabular}{|>{\raggedright}p{63pt}|>{\raggedright}p{21pt}|>{\raggedright}p{25pt}|>{\raggedright}p{21pt}|>{\raggedright}p{21pt}|>{\raggedright}p{21pt}|>{\raggedright}p{21pt}|>{\raggedright}p{21pt}|>{\raggedright}p{24pt}|}
\hline
\multicolumn{9}{|p{243pt}|}{Type conversions}\tabularnewline
\hline
 & atom & integer & float & pid & fun & tuple & list & binary\tabularnewline
\hline
atom &  & - & - & - & - & - & X & X\tabularnewline
\hline
integer & - &  & X & - & - & - & X & X\tabularnewline
\hline
float & - & X &  & - & - & - & X & X\tabularnewline
\hline
pid & - & - & - &  & - & - & X & X\tabularnewline
\hline
fun & - & - & - & - &  & - & X & X\tabularnewline
\hline
tuple & - & - & - & - & - &  & X & X\tabularnewline
\hline
list & X & X & X & X & X & X &  & X\tabularnewline
\hline
binary & X & X & X & X & X & X & X & \tabularnewline
\hline
\end{tabular}
\end{center}

The BIF \texttt{float/1} converts integers to floats. The BIFs
\texttt{round/1} and \texttt{trunc/1} convert floats to integers.

The BIFs \texttt{Type\_to\_list/1} and \texttt{list\_to\_Type/1}
convert to and from lists.

The BIFs \texttt{term\_to\_binary/1} and \texttt{binary\_to\_term/1}
convert to and from binaries.

\begin{erlang}
atom_to_list(hello)        % "hello"
list_to_atom("hello")      % hello
float_to_list(7.0)         % "7.00000000000000000000e+00"
list_to_float("7.000e+00") % 7.00000
integer_to_list(77)        % "77"
list_to_integer("77")      % 77
tuple_to_list({a, b ,c})   % [a,b,c]
list_to_tuple([a, b, c])   % {a,b,c}
pid_to_list(self())        % "<0.25.0>"
term_to_binary(<<17>>)     % <<131,109,0,0,0,1,17>>
term_to_binary({a, b ,c})  % <<131,104,3,100,0,1,97,100,0,1,98,100,0,1,99>>
binary_to_term(<<131,104,3,100,0,1,97,100,0,1,98,100,0,1,99>>)  % {a,b,c}
term_to_binary(math:pi())  % <<131,99,51,46,49,52,49,53,57,50,54,53,51,...>>
\end{erlang}

